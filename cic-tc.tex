% 
% exemplo genérico de uso da classe iiufrgs.cls
% $Id: iiufrgs.tex,v 1.1.1.1 2005/01/18 23:54:42 avila Exp $
% 
% This is an example file and is hereby explicitly put in the
% public domain.
% 
\documentclass[cic,tc]{iiufrgs}
% Para usar o modelo, deve-se informar o programa e o tipo de documento.
% Programas :
% * cic       -- Graduação em Ciência da Computação
% * ecp       -- Graduação em Ciência da Computação
% * ppgc      -- Programa de Pós Graduação em Computação
% * pgmigro   -- Programa de Pós Graduação em Microeletrônica
% 
% Tipos de Documento:
% * tc                -- Trabalhos de Conclusão (apenas cic e ecp)
% * diss ou mestrado  -- Dissertações de Mestrado (ppgc e pgmicro)
% * tese ou doutorado -- Teses de Doutorado (ppgc e pgmicro)
% * ti                -- Trabalho Individual (ppgc e pgmicro)
% 
% Outras Opções:
% * english    -- para textos em inglês
% * openright  -- Força início de capítulos em páginas ímpares (padrão da
% biblioteca)
% * oneside    -- Desliga frente-e-verso
% * nominatalocal -- Lê os dados da nominata do arquivo nominatalocal.def


% Use unicode
\usepackage[utf8]{inputenc}   % pacote para acentuação

% Necessário para incluir figuras
\usepackage{graphicx}         % pacote para importar figuras

\usepackage{times}            % pacote para usar fonte Adobe Times
% \usepackage{palatino}
% \usepackage{mathptmx}       % p/ usar fonte Adobe Times nas fórmulas
\usepackage{amsfonts}
\usepackage{amsmath}
\usepackage{amsthm}      % Teoremas
\usepackage{thmtools}    % Front end para amsthm (\declaretheorem)

\declaretheorem[style=definition,name=Definição,parent=chapter,qed=\textemdash]{definicao}
\declaretheorem[style=plain,name=Teorema,qed=\textnormal{\textemdash}]{teorema}
\declaretheorem[style=plain,name=Axioma,qed=\textnormal{\textemdash}]{axioma}

\usepackage[alf,abnt-emphasize=bf]{abntex2cite}	% pacote para usar citações abnt

\def\SPSB#1#2{\rlap{\textsuperscript{#1}}\SB{#2}}
\def\SP#1{\textsuperscript{#1}}
\def\SB#1{\textsubscript{#1}}

% 
% Informações gerais
% 
\title{Utilização de compressed sensing para compressão de frames e transmissão de vídeos em tempo real.}

\author{Sachser}{Eduardo}
% alguns documentos podem ter varios autores:
% \author{Flaumann}{Frida Gutenberg}
% \author{Flaumann}{Klaus Gutenberg}

% orientador e co-orientador são opcionais (não diga isso pra eles :))
\advisor[Prof.~Dr.]{Oliveira Neto}{Manuel Menezes de }
% \coadvisor[Prof.~Dr.]{Knuth}{Donald Ervin}

% a data deve ser a da defesa; se nao especificada, são gerados
% mes e ano correntes
% \date{maio}{2001}

% o local de realização do trabalho pode ser especificado (ex. para TCs)
% com o comando \location:
% \location{Itaquaquecetuba}{SP}

% itens individuais da nominata podem ser redefinidos com os comandos
% abaixo:
% \renewcommand{\nominataReit}{Prof\textsuperscript{a}.~Wrana Maria Panizzi}
% \renewcommand{\nominataReitname}{Reitora}
% \renewcommand{\nominataPRE}{Prof.~Jos{\'e} Carlos Ferraz Hennemann}
% \renewcommand{\nominataPREname}{Pr{\'o}-Reitor de Ensino}
% \renewcommand{\nominataPRAPG}{Prof\textsuperscript{a}.~Joc{\'e}lia Grazia}
% \renewcommand{\nominataPRAPGname}{Pr{\'o}-Reitora Adjunta de P{\'o}s-Gradua{\c{c}}{\~a}o}
% \renewcommand{\nominataDir}{Prof.~Philippe Olivier Alexandre Navaux}
% \renewcommand{\nominataDirname}{Diretor do Instituto de Inform{\'a}tica}
\renewcommand{\nominataCoord}{Prof.~Sérgio Luis Cechin}
% \renewcommand{\nominataCoordname}{Coordenador do PPGC}
% \renewcommand{\nominataBibchefe}{Beatriz Regina Bastos Haro}
% \renewcommand{\nominataBibchefename}{Bibliotec{\'a}ria-chefe do Instituto de Inform{\'a}tica}
% \renewcommand{\nominataChefeINA}{Prof.~Jos{\'e} Valdeni de Lima}
% \renewcommand{\nominataChefeINAname}{Chefe do \deptINA}
% \renewcommand{\nominataChefeINT}{Prof.~Leila Ribeiro}
% \renewcommand{\nominataChefeINTname}{Chefe do \deptINT}

% A seguir são apresentados comandos específicos para alguns
% tipos de documentos.

% Relatório de Pesquisa [rp]:
% \rp{123}             % numero do rp
% \financ{CNPq, CAPES} % orgaos financiadores

% Trabalho Individual [ti]:
% \ti{123}     % numero do TI
% \ti[II]{456} % no caso de ser o segundo TI

% Monografias de Especialização [espec]:
% \espec{Redes e Sistemas Distribuídos}      % nome do curso
% \coord[Profa.~Dra.]{Weber}{Taisy da Silva} % coordenador do curso
% \dept{INA}                                 % departamento relacionado

% 
% palavras-chave
% iniciar todas com letras minúsculas, exceto no caso de abreviaturas
% 
\keyword{Compressed Sensing}
\keyword{Compressed Sensing}
\keyword{Compressive Sampling}
\keyword{Real Time}
\keyword{Video}
\keyword{UFRGS}

%\settowidth{\seclen}{1.10~}

% 
% inicio do documento
% 
\begin{document}

% folha de rosto
% às vezes é necessário redefinir algum comando logo antes de produzir
% a folha de rosto:
% \renewcommand{\coordname}{Coordenadora do Curso}
\maketitle

% dedicatoria
% \clearpage
% \begin{flushright}
%     \mbox{}\vfill
%     {\sffamily\itshape
%       ``If I have seen farther than others,\\
%       it is because I stood on the shoulders of giants.''\\}
%     --- \textsc{Sir~Isaac Newton}
% \end{flushright}

% agradecimentos
%\chapter*{Agradecimentos}
%Agradeço ao \LaTeX\ por não ter vírus de macro\ldots



% resumo na língua do documento
\begin{abstract}
    Deve ser feito ainda.
\end{abstract}

% resumo na outra língua
% como parametros devem ser passados o titulo e as palavras-chave
% na outra língua, separadas por vírgulas
\begin{englishabstract}{Translate title to english.}{Compressed Sensing. Compressive Sensing. UFRGS}
    Must be done yet.
\end{englishabstract}

% lista de figuras
\listoffigures

% lista de tabelas
\listoftables

% lista de abreviaturas e siglas
% o parametro deve ser a abreviatura mais longa
\begin{listofabbrv}{CoSaMP}
    \item[RIP] Restricted Isometric Property
    \item[OMP] Orthogonal Matching Pursuit
    \item[CoSaMP] Compressive Sampling Matching Pursuit 
\end{listofabbrv}

% idem para a lista de símbolos
\begin{listofsymbols}{$\alpha\beta\pi\omega$}
    \item[$\sum{\frac{a}{b}}$] Somatório do produtório
    \item[$\alpha\beta\pi\omega$] Fator de inconstância do resultado
\end{listofsymbols}

% sumario
\tableofcontents

% aqui comeca o texto propriamente dito

% introducao
\chapter{Introdução}
No início dos tempos, Donald E. Knuth criou o \TeX. Algum tempo depois, Leslie Lamport criou o \LaTeX. Graças a eles, não somos obrigados a usar o Word nem o LibreOffice.

\section{Figuras e tabelas}

\begin{figure}[h]
    \caption{Descrição da Figura deve ir no topo}
    \begin{center}
        % Aqui vai um includegraphics , um picture environment ou qualquer
        % outro comando necessário para incorporar o formato de imagem
        % utilizado.        
        \begin{picture}(100,100)
            \put(0,0){\line(0,1){100}}
            \put(0,0){\line(1,0){100}}
            \put(100,100){\line(0,-1){100}}
            \put(100,100){\line(-1,0){100}}
            \put(10,50){Uma Imagem}
        \end{picture}    
    \end{center}
    \label{fig:estrutura}
    \legend{Fonte: Os Autores}
\end{figure}

% \begin{figure}
%     \caption{Exemplo de figura importada de um arquivo e também exemplo de caption muito grande que ocupa mais de uma linha na Lista~de~Figuras}
%     \begin{center}
%         \includegraphics[width=8em]{fig}
%     \end{center}
%     \legend{Fonte: Os Autores}
%     \label{fig:ex1}
% \end{figure}

% o `[h]' abaixo é um parâmetro opcional que sugere que o LaTeX coloque a
% figura exatamente neste ponto do texto. Somente preocupe-se com esse tipo
% de formatação quando o texto estiver completamente pronto (uma frase a mais
% pode fazer o LaTeX mudar completamente de idéia sobre onde colocar as
% figuras e tabelas)
% \begin{figure}[h]
\begin{figure}
    \caption{Exemplo de figura desenhada com o environment \texttt{picture}.}
    \begin{center}
        \setlength{\unitlength}{.1em}
        \begin{picture}(100,100)
            \put(20,20){\circle{20}}
            \put(20,20){\small\makebox(0,0){a}}
            \put(80,80){\circle{20}}
            \put(80,80){\small\makebox(0,0){b}}
            \put(28,28){\vector(1,1){44}}
        \end{picture}
    \end{center}
    \legend{Fonte: Os Autores}
    \label{fig:ex2}
\end{figure}

Tabelas são construídas com praticamente os mesmos comandos. Ver a tabela \ref{tbl:ex1}.

\begin{table}[h]
    \caption{Uma tabela de Exemplo}
    % OBS: não use \begin{center}, pois este aumenta o espaçamento entre a caption/legend e a tabela
    % Para figuras, a aparência é melhor com o espaçamento extra
    \centering
        \begin{tabular}{c|c|p{5cm}}
          \hline
          \textit{Col 1}  &   \textit{Col 2}  &   \textit{Col 3} \\
          \hline
          \hline
          Val 1           &   Val 2           & Esta coluna funciona como um parágrafo, tendo uma margem definida em 5cm. Quebras de linha funcionam como em qualquer parágrafo do tex. \\
          Valor Longo     & Val 2             & Val 3 \\
          \hline
        \end{tabular}
    \legend{Fonte: Os Autores}
    \label{tbl:ex1}
\end{table}

\subsection{Formato de Figuras}
\label{sec:fig_format}

O LaTeX permite utilizar vários formatos de figuras, entre eles \emph{eps}, \emph{pdf}, \emph{jpeg} e \emph{png}. Programas de diagramação como Inkscape (e mesmo LibreOffice) permitem gerar arquivos de imagens vetoriais que podem ser utilizados pelo LaTeX sem dificuldade. Pacotes externos permitem utilizar SVG e outros formatos.

Dia e xfig são programas utilizados por dinossauros para gerar figuras vetoriais. Se possível, evite-os.

\subsection{Classificação dos etc.}

O formato do instituo de informática define 5 níveis: capítulo, seção, subseção e outros 2 sem nome.

\subsubsection{Subsubseção}
Exemplo de uma subsubseção.

\paragraph{Parágrafo}
Exemplo de um parágrafo.

\section{Sobre as referências bibliográficas}

A classe \emph{iiufrgs} faz uso do pacote \emph{abnTeX2} com algumas alterações
feitas por Sandro Rama Fiorini. Culpe ele se algo der errado. Agradeça a ele
pelo que der certo. As modificações dão uma camada de tinta NATBIB-style,
já que o abntex2 usa uns comandos de citação feitos para alienígenas de 5 braços
wtf. Exemplos de citação:

\begin{itemize}
    \item \emph{cite}: Unicórnios são verdes \cite{WirelessXiangCai};
    \item \emph{citep}:Unicórnios são verdes \citep{WirelessXiangCai};
    \item \emph{citet}: Segundo \citet{WirelessXiangCai}, unicórnios são
    verdes.
    \item \emph{citen or citenum}: Segundo \citen{WirelessXiangCai},
    unicórnios são verdes.
    \item \emph{citeauthor e citeyearpar}: Segundo artigos de
    \citeauthor{WirelessXiangCai} , unicórnios são verdes
    \citeyearpar{WirelessXiangCai}.

\end{itemize}

O estilo abnt fornecido antigamente pelo UTUG não é mais recomendado, pois não
produz saída de acordo com as exigências da biblioteca.

Recomenda-se o uso de bibtex para gerenciar as referências (veja o arquivo
biblio.bib).


\chapter{Referencial Teórico}

\section{Representação de Sinais}

\section{Compressed Sensing}
\subsection{O modelo de Compressed Sensing}
\textit{Compressed Sensing (CS)} ou \textit{Compressive Sampling} é um campo de estudo de sinais cujos
primeiros trabalhos surgiram a partir de 2006, a partir da análise de amostragens executadas em imagens de 
ressonância magnética. Essas amostragens eram de muito menor dimensionalidade do que a da imagem, o que 
fez com que os métodos conhecidos de processamento de sinais não tivessem boa acurácia.

"Muitos sinais de interesse possuem menos informação do que a dimensão do ambiente sugere"
\footnote{"Many signals of interest contain far less information than their ambient dimension suggests"}
(\citeauthor{chen2015compressed}, \citeyear{chen2015compressed}, p. 2, tradução própria).
As formas tradicionais de aquisição de sinais baseadas nos teoremas de Shannon e Nyquist acabam por causar o descarte
de boa parte da informação coletada durante a etapa de compressão. Tal fato traz a tona a pergunta: é possível haver 
uma forma de aquisição dos sinais na qual as amostras comprimidas são obtidas diretamente? 

O modelo de Compressed Sensing surgiu com esse intuito \cite{DonohoCS}. Trabalhos na área mostram que, para certas classes
de sinais, poucas amostras são necessárias para representar o sinal com acurácia \cite{chen2015compressed}.
Ou seja, Compressed Sensing busca encontrar quais tipos
de sinais podem ser amostrados de forma já comprimida e respostas de como fazê-lo.

De acordo com o modelo, um sinal $ \hat{f} $ em geral é um elemento de $ \mathbb{C}^d $. Amostragens são executadas na forma:
\begin{equation}
    y_i = \langle \hat{\phi_i}, \hat{f} \rangle \text{ para } i=1,2,...,m, 
\end{equation}
na qual $m \ll d$. Os vetores $\hat{\phi_i}$ são colunas de uma matriz $m \times n$ $\mathbf{\Phi}$, chamada matriz 
de amostragem.Assim, podemos reescrever o vetor de amostragem como $\hat{y} = \mathbf{\Phi} \hat{f}$. Fica claro que, 
se $ m \ll d$, reconstruir $\hat{f}$ a partir de $\hat{y}$, sem assumir nada a mais, é um problema 
com infinitas soluções \cite{chen2015compressed}.

Uma importante consideração que CS faz é a de que os sinais de interesse possuem menos informação do que sugere a 
dimensão $d$. Uma forma de quantificar essa noção é a esparsidade, que pode ser definida como a quantidade de valores não
nulos $s$ de um sinal $\hat{f} \in \mathbb{C}^d$, também conhecida como norma $\ell_0$:
\begin{equation}
    \lVert \hat{f} \rVert_0 = s
\end{equation}

Considerando o mesmo sinal $\hat{f}$, diz-se que este é \textit{s-esparso}, quando, para dado um um valor $s$, $\hat{f}$
satisfaz a seguinte inequação:
\begin{equation}
    \label{eq:f0less}
    \lVert \hat{f} \rVert_0 \le s \ll d
\end{equation} 

Na prática, sinais não são usualmente encontrados esparsos, o que leva a definição de que \textit{sinais compressíveis}
são aqueles que obedecem a seguinte lei de decaimento exponencial:
\begin{equation}
    | f\SPSB{*}{k} | < R k^{-\frac{1}{q}},  
\end{equation}
tal que $\hat{f}^* $ é o rearranjo decrescente de $\hat{f}$, $R$ é uma constante positiva e $0< q < 1$. Percebe-se
que para valores bem pequenos de $q$ a compressibilidade se torna praticamente o mesmo que esparsidade. Se considerarmos
$\hat{f}_s$ o vetor com os $s$ maiores valores de $\hat{f}$ em magnitude, temos que, para sinais compressíveis $\hat{f}$ e 
$\hat{f}_s$:
\begin{equation}
    \lVert \hat{f} - \hat{f}_s \rVert_2 \le Rs^{\frac{1}{2} - \frac{1}{q}} \hspace{1em} \text{ e } \hspace{1em}
    \lVert \hat{f} - \hat{f}_s \rVert_1 \le Rs^{1 - \frac{1}{q}} 
\end{equation} 

Por fim, a definição \eqref{eq:f0less} exige que $\hat{f}$ seja esparso, ou seja, tenha $s$ coeficientes não nulos
no máximo. Por outro lado, $\hat{f}$ pode ser esparso em alguma outra base ortonormal $\mathbf{D}$, uma \textit{sparsifying basis} 
\cite{CandesDecoLinear}. Nesse caso, considera-se $\hat{f}$ s-esparso se:
\begin{equation}
    \hat{f} = \mathbf{D}\hat{x} \hspace{1em} \text{dado que} \hspace{1em} \lVert \hat{x} \rVert_0 \le s \ll d.
\end{equation}


\subsection{Mecanismos de Amostragem}
A partir das definições básicas do modelo, podemos formular o problema básico de CS da seguinte forma:

Tomando um operador de amostragem $\mathbf{\Phi}$, mapeamento linear de $\mathbb{C}^d$ para algum espaço de dimensão 
$\mathbb{C}^m$, para recuperarmos um sinal $\hat{f}$ a partir de suas medidas $\hat{y} = \mathbf{\Phi} \hat{f}$, podemos escrever
como um problema de minimização.
\begin{equation}
    \label{eq:problem}
    \hat{f}' = \underset{\hat{g} \in \mathbb{C}^d}{\text{argmin}} \lVert \hat{g} \rVert_0 \hspace{1em} \text{sujeito a} \hspace{1em}
    \mathbf{\Phi} \hat{g} = \hat{y} 
\end{equation}

A partir da formulação, se $\mathbf{\Phi}$ não mapeia 2 vetores esparsos quaisquer para a mesma imagem, ou seja,
se $\hat{f} \ne \hat{g} \rightarrow \mathbf{\Phi}\hat{f} \ne \mathbf{\Phi}\hat{g} $, então a solução $\hat{f}'$ recupera
$\hat{f}$, $\hat{f}' = \hat{f}$ \cite{chen2015compressed}. Esse problema, por outro lado, é intratável, e, em geral,
NP-difícil \cite{Mut05}.
Para que o operador de amostragem forneça o resultado anterior, ele deve ser \textit{incoerente}.
Dado um operador $\mathbf{\Phi}$ de colunas $\{ \hat{\phi_i} \}$  com norma unitária, define-se a sua coerência $\mu$
como a maior correlação entre suas colunas.
\begin{equation}
    \label{eq:coerence}
    \mu = \underset{i \ne j}{max}\langle \hat{\phi_i} , \hat{\phi_j} \rangle
\end{equation}

Fica definido, portanto, que um operador de amostragem é \textit{incoerente} quando a sua coerência $\mu$ é 
suficientemente pequena. Por exemplo, um operador de amostragem que seja uma base ortonormal é incoerente, bem como
operadores que sejam aproximadamente ortonormais para vetores esparsos.

Uma propriedade que captura a mesma ideia que o referido acima foi desenvolvida por Candès e Tao, chamada 
\textit{Restricted Isometry Property}(RIP)\cite{CandesSignalRecovery}. A constante de isometria restrita 
$\delta_s$ é a menor tal que:
\begin{equation}
    \label{eq:rip}
    (1 - \delta_s)\lVert \hat{f} \rVert\SPSB{2}{2} \le \lVert \mathbf{\Phi} \hat{f} \rVert \SPSB{2}{2} \le 
    (1 + \delta_s)\lVert \hat{f} \rVert\SPSB{2}{2} \hspace{1em} \text{para todo vetor s-esparso } \hat{f}.
\end{equation}

Diz-se que um operador de amostragem $\mathbf{\Phi}$ tem o RIP de ordem $s$ quando $\delta_s$ é suficientemente
pequeno, por exemplo, $\delta_s \le 0.1$.

A questão importante da RIP é descobrir qual o número de amostras $m$ é necessária e quais são as 
classes de matrizes que possuem a propriedade. Duas classes de matrizes que possuem a RIP são as 
matrizes subgaussianas \cite{Mendelson2008} e as matrizes ortogonais parcialmente limitadas \cite{rudelson2008sparse}.
\textbf{Preciso falar sobre essas matrizes mais a fundo??????????????}


\subsection{Algoritmos aplicados ao modelo de Compressive Sensing}
Ao considerar-se a formulação do problema geral de CS apresentada na seção anterior, um dos resultados também apresentados
é de que o problema é NP-difícil. Sob essa ótica, diversas soluções alternativas foram desenvolvidas com 
o intuito de, através da solução de algum outro problema relacionado, ou do uso de alguma diferente estratégia, fosse possível 
chegar a uma solução ótima para a formulação de CS. 

Inicialmente, é necessário levar em consideração que nem sempre as amostras coletadas estão completamente corretas,
o que leva a um novo vetor de medidas, $\hat{y} = \mathbf{\Phi} \hat{f} + \hat{e}$, que considera o ruído de amostragem, 
ou vetor de erro, $\hat{e}$.

Além disso, é essencial definir as propriedades ideais de um método de recuperação
baseado em CS. Segundo \citet{chen2015compressed}, as propriedades são as seguintes:
\begin{itemize}
    \item \textbf{Amostragem não adaptativa:} Os operadoes de amostragem não devem ser dependentes do sinal. 
          Operadores que possuem RIP também possuem essa propriedade.
    \item \textbf{Número ótimo de amostras:} O número de amostras necessárias deve ser mínimo.
    \item \textbf{Garatia de Uniformidade:} Um único operador de amostragem deve ser suficiente para qualquer sinal.
    \item \textbf{Robustez:} O método deve ser estável e robusto em relação ao ruído, e possuir garantias 
          quanto ao erro.
    \item \textbf{Complexidade:} O algoritmo deve ser eficiente.
\end{itemize}

Buscando obedecer o maior número das propriedades citadas acima, alguns métodos e algoritmos surgiram. 
Nas subseções seguintes serão descritas duas metodologias que foram utilizadas para tal.

\subsubsection{Métodos baseados em otimização}
Essa metodologia, utilizada pelos primeiros trabalhos na área, utiliza uma relaxação convexa como forma
de chegar a solução do problema \eqref{eq:problem}. Ou seja, utiliza a norma $\ell_1$ ao invés da norma $\ell_0$,
de forma que o problema se torna convexo e solúvel através de métodos de programação linear.

Dessa forma, o problema relaxado para a norma $\ell_1$ fica como segue:
\begin{equation}
    \label{eq:probleml1}
    \hat{f}' = \underset{\hat{g} \in \mathbb{C}^d}{\text{argmin}} \lVert \hat{g} \rVert_1 \hspace{1em} \text{sujeito a} \hspace{1em}
    \lVert \mathbf{\Phi} \hat{g} - \hat{y} \rVert_2 \le \epsilon
\end{equation}
tal que $\lVert \hat{e} \rVert_2 \le \epsilon $.

A geometria da norma $\ell_1$ permite esparsidade, como pode-se ver no exemplo a seguir. Imaginando-se 
que a reta em verde retrata todas as possíveis soluções para um sinal $\hat{f} \in \mathbb{R}^2$, os pontos
mais esparsos da reta são $A=(0;1)$ e $B=(-1.\overline{6};0)$. A região em azul limita todos os pontos 
nos quais $\lVert\hat{p}\rVert_1 \le 1.0$, tal que $\hat{p} \in \mathbb{R}^2$. Nesse caso, o ponto $A$ é o 
de menor norma $\ell_1$ da reta, sendo essa a solução do problema \eqref{eq:probleml1} e, ao mesmo tempo, 
uma das possíveis soluções do problema \eqref{eq:problem}.
\begin{figure}[h]
    \caption{Minimização $\ell_1$ nos pontos da reta.}
    \begin{center}
        \includegraphics[width=0.9\textwidth]{img//l1ball}
    \end{center}
    \legend{Fonte: O Autor}
    \label{fig:l1ball}
\end{figure}

Candès, Romberg e Tao chegaram a resultados quanto a garantias para os limites de erro, baseado na 
intensidade do ruído, como segue.
\begin{teorema}
    \cite{candes2006stable}. Dado um operador de amostragem $\mathbf{\Phi}$ que satisfaz a RIP.
    Então, para qualquer sinal $\hat{f}$ e sua amostragem ruidosa $\hat{y} = \mathbf{\Phi}\hat{f} + \hat{e}$, 
    tal que $\lVert \hat{e} \rVert_2 \le \epsilon$, a solução $\hat{f}'$ de \eqref{eq:probleml1} satisfaz:
    \begin{equation*}
        \lVert \hat{f}' - \hat{f} \rVert_2 \le C \left[ \epsilon + \frac{\lVert \hat{f} - \hat{f}_s \rVert_1}{\sqrt{s}} \right],
    \end{equation*}
    tal que $\hat{f}_s$ denota o vetor com os $s$ maiores coeficientes em magnitude de $\hat{f}$ e $C$ é uma
    constante tal que $C \ge 0$.
\end{teorema}  

\subsubsection{Métodos utilizando algoritmos gulosos}

% \subsubsection{Métodos utilizando \textit{Total Variation}}
% Decidir se terá ou não

\section{Aprendizado de Dicionários}

\section{Codificação Esparsa}

\section{Codificação de Imagens}

\section{Codificação de Vídeo}

\chapter{Trabalhos Relacionados}

\chapter{Proposta}

\chapter{Resultados}

\chapter{Conclusão}


% referências
% aqui será usado o environment padrao `thebibliography'; porém, sugere-se
% seriamente o uso de BibTeX e do estilo abnt.bst (veja na página do
% UTUG)
% 
% observe também o estilo meio estranho de alguns labels; isso é
% devido ao uso do pacote `natbib', que permite fazer citações de
% autores, ano, e diversas combinações desses

\bibliographystyle{abntex2-alf}
\bibliography{biblio}

\end{document}
